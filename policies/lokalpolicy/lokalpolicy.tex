\documentclass[11pt, includeaddress]{classes/cthit}
\usepackage{titlesec}

\titleformat{\paragraph}[hang]{\normalfont\normalsize\bfseries}{\theparagraph}{1em}{}
\titlespacing*{\paragraph}{0pt}{3.25ex plus 1ex minus 0.2ex}{0.7em}

\graphicspath{ {images/} }

\begin{document}

\title{Lokalpolicy}
\approved{2013--02--28}
%\revisioned{2016--12--08}
%\revisioned{2018--11--11}
%\revisioned{2020--02--27}
%\revisioned{2020--05--14}
\revisioned{2022--10--06}

\maketitle

\thispagestyle{empty}

\newpage

\makeheadfoot%

%Rubriksnivådjup
\setcounter{tocdepth}{2}
%Sidnumreringsstart
\setcounter{page}{1}
\tableofcontents

\newpage

%MYhref definerad i classen, ger möjlighet att byta färg mm


\section{Formalia}
Med sektionen avses Teknologsektionen Informationsteknik.

\subsection{Hänvisningar}
Förutom IT­-sektionens egen lokalpolicy, lyder också sektionens medlemmar under kårens centrala lokalpolicy samt dispositionsavtalet mot Chalmers
Tekniska Högskola AB. Se nedan:
\newline
\MYhref{https://chalmersstudentkar.se/wp-content/uploads/2017/07/Lokalpolicy.pdf}{Kårens lokalpolicy}\\
\MYhref{https://chalmers.it/api/media/dj8HJezcQfrqZsHFjkkX212kS5SZ-5mZq0clOS58n2Y}{Dispositionsavtal}

\subsection{Dokumentets Syfte}
Följande dokument beskriver vilka riktlinjer och regler som gäller för de lokaler som nyttjas av
sektionen. Dokumentet beskriver vilka möjligheter som finns för sektionens medlemmar vad gäller
bokning av dessa lokaler, utöver vilka förhållningsregler som finns i respektive lokal.

\section{Förteckning}
Sektionens lokaler består av:
\begin{itemize}
  \item \HUBBEN\
  	\begin{itemize}
	  \item Stora rummet
	  \item Köket
	  \item Grupprummet
	  \item Studierummet
	  \item Föreningsrummet
	\end{itemize}
   \item HASen
\end{itemize}


\section{Hubben}
Lokalerna som tillhör \HUBBEN\ är till för medlemmar av sektionen. \HUBBEN\ är i första hand avsedd för
studier och som lunchlokal. Det bör hållas en rimlig ljudnivå under studietid (08.00­ - 17.00 på
studiedagar). Under den tiden får ingen alkohol förtäras i lokalerna. Vidare är \HUBBEN\ även avsedd för
arrangemang som gynnar och välkomnar sektionens medlemmar och skall vara en plats där alla
medlemmar kan umgås.

\subsection{StorHubben}
StorHubben är den enda delen av \HUBBEN\ som täcks av serveringstillstånd vid arrangemang med sådant.

\subsection{Grupprummet}
Grupprummet är främst avsett för individuella studier och grupparbeten.

\subsection{Chalmers Tekniska Chillhörna (CTC)}
CTC är främst avsett för individuella studier och grupparbeten.

\subsection{Studierummet}
Studierummet är avsett för studier och ska alltid vara tillgängligt för samtliga medlemmar av sektionen (undantag kan göras vid större arrangemang eller vid godkännande från \STYRIT{}). Här ska det hållas en låg ljudnivå.
\newline
I Studierummet råder alltid strikt fest- och alkoholförbud.

\subsection{Föreningsrummet}
Föreningsrummet är endast avsett som förvaringsutrymme för sektionens aktiva organ samt föreningar.


\section{HASen}
HASen är avsedd som förrådsrum för sektionens aktiva organ samt föreningar.

HASen skall på grund av sitt läge alltid vara låst. I HASen råder strikt festförbud på samma vis som i studierummet.

\section{Tillträde}
Alla medlemmar av sektionen har alltid tillträde till \HUBBEN\ såvida ej serveringstillstånd gäller eller
om arrangör i lokalen satt upp särskilda regler som godkänts av \STYRIT. I undantagsfall har \STYRIT{} rätt
att frånta enskild teknolog tillträde till \HUBBEN{}. Under mottagningsperioden har även IT-sektionens mottagningskommitté rätt att, i samrådan med \STYRIT{}, upprätta regler kring vilka som får besöka \HUBBEN{} under mottagningsarrangemang.


Det är inte under några omständigheter tillåtet att sova i någon av sektionens lokaler.


\section{Hyra}
\subsection{Lokaler}
\HUBBEN\ får ej hyras ut. \HUBBEN\ får däremot lånas ut inom Chalmers Studentkår om goda skäl
föreligger och görs i sådana fall i samtycke med ansvarig kommitté.

\subsection{Inventarier}
Hubbens inventarier får hyras ut för lämplig avgift i samband med att \HUBBEN\ lånas ut till extern part.
Vidare får Hubbens inventarier ej lämna Hubben utan tillstånd från ansvarig kommitté.


\section{Boka}
\HUBBEN{}s olika lokaler får bokas efter kl 17:00 på studiedagar samt dygnet runt på helger. 
Undantag gäller från tentaperioders första dag kl 08.00 till dess sista dag kl 13.00, då \HUBBEN\ inte får bokas annat än till studierelaterade arrangemang.
Utöver detta får sektionens lokaler bara bokas 9 veckor fram i tiden. 
Styrelsen samt lokalansvarig kommitté har rätt att häva bokningar om de ej anses lämpliga.
Undantag gäller under mottagningen då mottagningskommittén i samrådan med styrelsen samt lokalansvarig kommitté är ansvariga för bokningar av \HUBBEN{}. 
Det kan förekomma undantag från dessa regler om de anses lämpliga och fått styrelsen godkännande.

\subsection{StorHubben}
StorHubben får bokas av medlemmar av sektionens aktiva organ och föreningar.
Dessa bokningar får endast göras för sektionsnyttiga ändamål.

\subsection{Grupprummet}
Grupprummet får bokas enligt §7.

\subsection{Studierummet}
Studierummet får ej bokas.

\subsection{Chalmers Tekniska Chillhörna (CTC)}
CTC får bokas enligt §7.


\section{Inventarier}
Hubbens inventarier får nyttjas gratis av medlemmar av sektionen vid godkännande av ansvarig kommitté. styrIT och Chalmers säkerhetssamordnare skall
konsulteras innan inventarier som kan påverka brandsäkerheten negativt införskaffas. Hubbens inventarier får ej lämna lokalen utan medgivande från
ansvarig kommitté.



\section{Vård av lokaler}
Det skall ej förekomma nedskräpning eller onödigt slitage på sektionens lokaler.
Man skall vidare som medlem av sektionen behandla sektionens lokaler med varsamhet.

\section{Ansvar vid arrangemang}
Arrangör är skyldig att vara väl införstådd med detta dokument, lokalens brand­, och
utrymmningsplan, lokalens dispositionsavtal samt studentkårens lokalpolicy.
Arrangör är vidare ansvarig för
\begin{itemize}
	\item Att skador som uppstått på lokalen under arrangemang ersätts.
	\item Att återställa lokalen till ett gott skick efter arrangemang.
	\item Att antalet person som vistas i lokalen ej överskrider den i dispositionsavtalet specificerade
	gränsen.
	\begin{itemize}
	\item Antalet skall inkludera alla som befinner sig innanför Hubben­dörren.
	\end{itemize}
\end{itemize}


\end{document}
