\documentclass[8pt, includeaddress]{classes/cthit}
\usepackage{geometry}
\usepackage{titlesec}
\usepackage{verbatimbox}
\usepackage{tabularx}
\usepackage{hyperref}

\usepackage{pgfkeys}

\geometry{a4paper, portrait, left=0.5cm, right=0.5cm}

% Set up the keys.  Only the ones directly under /mytextfield
% can be accepted as options to the \mytextfield macro.
\pgfkeys{
 /mytextfield/.is family, /mytextfield,
 % Here are the options that a user can pass
 default/.style = 
  {borderwidth = 0, dotwidth = 3cm, name=herp},
 borderwidth/.estore in = \mytextfieldBorderwidth,
 dotwidth/.estore in = \mytextfieldDotwidth,
 name/.estore in = \mytextfieldName,
}

\newdimen\longline
\longline=\textwidth\advance\longline-6cm
\def\LayoutTextField#1#2{#2} % override default in hyperref

\def\lbl#1{\hbox to \mytextfieldDotwidth{#1\dotfill\strut}}%
\def\labelline#1#2{\lbl{#1}\vbox{\hbox{\TextField[name=\mytextfieldName,width=#2, borderwidth=\mytextfieldBorderwidth]{\null}}\kern2pt\hrule}}

% We process the options first, then pass them to `\parbox` in the form of macros.
\newcommand\mytextfield[2][]{%
 \pgfkeys{/mytextfield, default, #1}%
 \hbox to \hsize{\labelline{#2}{\longline}}\vskip1.4ex
}

\titleformat{\paragraph}[hang]{\normalfont\normalsize\bfseries}{\theparagraph}{1em}{}
\titlespacing*{\paragraph}{0pt}{1ex plus 1ex minus 0.2ex}{0.7em}

\graphicspath{ {images/} }

\title{Namnlista}
\makeheadfoot%

\begin{document}
  \begin{Form}

\section*{}
\mytextfield[name=namn1]{Datum:}{}

Jag som skrivit under önskar att Sektionsmötet diskuterar och beslutar om Hedersmedlemsskap i Teknologsektionen Informationsteknik för <namn>.

\setlength{\extrarowheight}{12pt}

\begin{tabular}{ | p{0.5cm} | p{8cm} | p{8cm} |}
	\multicolumn{1}{l}{\large{\textbf{}}} & 
	\multicolumn{1}{l}{\large{\textbf{Namn}}} & 
	\multicolumn{1}{l}{\large{\textbf{Signatur}}} \\
	\hline
	{1} &
	{\TextField[name=art1, borderwidth=0,width=8cm]{ }} & 
	{\TextField[name=medlem1, borderwidth=0,width=8cm]{ }} \\
	\hline
	{2} &
	{\TextField[name=art2, borderwidth=0,width=8cm]{ }} &
	{\TextField[name=medlem2, borderwidth=0,width=8cm]{ }} \\
	\hline
	{3} &
	{\TextField[name=art3, borderwidth=0,width=8cm]{ }} &
	{\TextField[name=medlem3, borderwidth=0,width=8cm]{ }} \\
	\hline
	{4} &
	{\TextField[name=art4, borderwidth=0,width=8cm]{ }} &
	{\TextField[name=medlem4, borderwidth=0,width=8cm]{ }} \\
	\hline
	{5} &
	{\TextField[name=art5, borderwidth=0,width=8cm]{ }} &
	{\TextField[name=medlem5, borderwidth=0,width=8cm]{ }} \\
	\hline
	{6} &
	{\TextField[name=art6, borderwidth=0,width=8cm]{ }} &
	{\TextField[name=medlem6, borderwidth=0,width=8cm]{ }} \\
	\hline
	{7} &
	{\TextField[name=art7, borderwidth=0,width=8cm]{ }} &
	{\TextField[name=medlem7, borderwidth=0,width=8cm]{ }} \\
	\hline
	{8} &
	{\TextField[name=art8, borderwidth=0,width=8cm]{ }} &
	{\TextField[name=medlem8, borderwidth=0,width=8cm]{ }} \\
	\hline
	{9} &
	{\TextField[name=art9, borderwidth=0,width=8cm]{ }} &
	{\TextField[name=medlem9, borderwidth=0,width=8cm]{ }} \\
	\hline
	{10} &
	{\TextField[name=art10, borderwidth=0,width=8cm]{ }} &
	{\TextField[name=medlem10, borderwidth=0,width=8cm]{ }} \\
	\hline
	{11} &
	{\TextField[name=art11, borderwidth=0,width=8cm]{ }} &
	{\TextField[name=medlem11, borderwidth=0,width=8cm]{ }} \\
	\hline
	{12} &
	{\TextField[name=art12, borderwidth=0,width=8cm]{ }} &
	{\TextField[name=medlem12, borderwidth=0,width=8cm]{ }} \\
	\hline
	{13} &
	{\TextField[name=art13, borderwidth=0,width=8cm]{ }} &
	{\TextField[name=medlem13, borderwidth=0,width=8cm]{ }} \\
	\hline
	{14} &
	{\TextField[name=art14, borderwidth=0,width=8cm]{ }} &
	{\TextField[name=medlem14, borderwidth=0,width=8cm]{ }} \\
	\hline
	{15} &
	{\TextField[name=art15, borderwidth=0,width=8cm]{ }} &
	{\TextField[name=medlem15, borderwidth=0,width=8cm]{ }} \\
	\hline
	{16} &
	{\TextField[name=art16, borderwidth=0,width=8cm]{ }} &
	{\TextField[name=medlem16, borderwidth=0,width=8cm]{ }} \\
	\hline
	{17} &
	{\TextField[name=art17, borderwidth=0,width=8cm]{ }} &
	{\TextField[name=medlem17, borderwidth=0,width=8cm]{ }} \\
	\hline
	{18} &
	{\TextField[name=art18, borderwidth=0,width=8cm]{ }} &
	{\TextField[name=medlem18, borderwidth=0,width=8cm]{ }} \\
	\hline %%%
\end{tabular}

\begin{tabular}{ | p{0.5cm} | p{8cm} | p{8cm} | p{3cm} |}
	\hline
	{19} &
	{\TextField[name=art19, borderwidth=0,width=8cm]{ }} & 
	{\TextField[name=medlem19, borderwidth=0,width=8cm]{ }} \\
	\hline
	{20} &
	{\TextField[name=art20, borderwidth=0,width=8cm]{ }} &
	{\TextField[name=medlem20, borderwidth=0,width=8cm]{ }} \\
	\hline
	{21} &
	{\TextField[name=art21, borderwidth=0,width=8cm]{ }} &
	{\TextField[name=medlem21, borderwidth=0,width=8cm]{ }} \\
	\hline
	{22} &
	{\TextField[name=art22, borderwidth=0,width=8cm]{ }} &
	{\TextField[name=medlem22, borderwidth=0,width=8cm]{ }} \\
	\hline
	{23} &
	{\TextField[name=art23, borderwidth=0,width=8cm]{ }} &
	{\TextField[name=medlem23, borderwidth=0,width=8cm]{ }} \\
	\hline
	{24} &
	{\TextField[name=art24, borderwidth=0,width=8cm]{ }} &
	{\TextField[name=medlem24, borderwidth=0,width=8cm]{ }} \\
	\hline
	{25} &
	{\TextField[name=art25, borderwidth=0,width=8cm]{ }} &
	{\TextField[name=medlem25, borderwidth=0,width=8cm]{ }} \\
	\hline
	{26} &
	{\TextField[name=art26, borderwidth=0,width=8cm]{ }} &
	{\TextField[name=medlem26, borderwidth=0,width=8cm]{ }} \\
	\hline
	{27} &
	{\TextField[name=art27, borderwidth=0,width=8cm]{ }} &
	{\TextField[name=medlem27, borderwidth=0,width=8cm]{ }} \\
	\hline
	{28} &
	{\TextField[name=art28, borderwidth=0,width=8cm]{ }} &
	{\TextField[name=medlem28, borderwidth=0,width=8cm]{ }} \\
	\hline
	{29} &
	{\TextField[name=art29, borderwidth=0,width=8cm]{ }} &
	{\TextField[name=medlem29, borderwidth=0,width=8cm]{ }} \\
	\hline
	{30} &
	{\TextField[name=art30, borderwidth=0,width=8cm]{ }} &
	{\TextField[name=medlem30, borderwidth=0,width=8cm]{ }} \\
	\hline
	{31} &
	{\TextField[name=art31, borderwidth=0,width=8cm]{ }} &
	{\TextField[name=medlem31, borderwidth=0,width=8cm]{ }} \\
	\hline
	{32} &
	{\TextField[name=art32, borderwidth=0,width=8cm]{ }} &
	{\TextField[name=medlem32, borderwidth=0,width=8cm]{ }} \\
	\hline
	{33} &
	{\TextField[name=art33, borderwidth=0,width=8cm]{ }} &
	{\TextField[name=medlem33, borderwidth=0,width=8cm]{ }} \\
	\hline
	{34} &
	{\TextField[name=art34, borderwidth=0,width=8cm]{ }} &
	{\TextField[name=medlem34, borderwidth=0,width=8cm]{ }} \\
	\hline
	{35} &
	{\TextField[name=art35, borderwidth=0,width=8cm]{ }} &
	{\TextField[name=medlem35, borderwidth=0,width=8cm]{ }} \\
	\hline
	{36} &
	{\TextField[name=art36, borderwidth=0,width=8cm]{ }} &
	{\TextField[name=medlem36, borderwidth=0,width=8cm]{ }} \\
	\hline 
	{37} &
	{\TextField[name=art37, borderwidth=0,width=8cm]{ }} &
	{\TextField[name=medlem37, borderwidth=0,width=8cm]{ }} \\
	\hline
	{38} &
	{\TextField[name=art38, borderwidth=0,width=8cm]{ }} &
	{\TextField[name=medlem38, borderwidth=0,width=8cm]{ }} \\
	\hline
	{39} &
	{\TextField[name=art39, borderwidth=0,width=8cm]{ }} &
	{\TextField[name=medlem39, borderwidth=0,width=8cm]{ }} \\
	\hline
	{40} &
	{\TextField[name=art40, borderwidth=0,width=8cm]{ }} &
	{\TextField[name=medlem40, borderwidth=0,width=8cm]{ }} \\
	\hline 
	{41} &
	{\TextField[name=art41, borderwidth=0,width=8cm]{ }} &
	{\TextField[name=medlem41, borderwidth=0,width=8cm]{ }} \\
	\hline
	{42} &
	{\TextField[name=art42, borderwidth=0,width=8cm]{ }} &
	{\TextField[name=medlem42, borderwidth=0,width=8cm]{ }} \\
	\hline 
\end{tabular}

 \end{Form}
\end{document}
