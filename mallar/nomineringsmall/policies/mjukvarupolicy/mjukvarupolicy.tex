\documentclass[11pt, includeaddress]{classes/cthit}
\usepackage{titlesec}

\titleformat{\paragraph}[hang]{\normalfont\normalsize\bfseries}{\theparagraph}{1em}{}
\titlespacing*{\paragraph}{0pt}{3.25ex plus 1ex minus 0.2ex}{0.7em}

\graphicspath{ {images/} }

\begin{document}

\title{Mjukvarupolicy}
\approved{2014--09--10}
\editorial{2014--09--17}
\maketitle

\thispagestyle{empty}

\newpage

\makeheadfoot%

%Rubriksnivådjup
\setcounter{tocdepth}{2}
%Sidnumreringsstart
\setcounter{page}{1}
\tableofcontents

\newpage

\section{Syfte}
Denna policy syftar till att dels IT-studenten ska ha möjlighet att få ta del av mjukvaran som sektionen utvecklar och dels att sektionens arbete kan underlättas då gemene teknolog kan hitta fel i koden och ge förslag till förbättringar (i så kallade pull-requests). Utöver detta bör utvecklandet av ny mjukvara i första hand ges till sektionens medlemmar så de kan få nyttig utvecklarerfarenhet.

\section{Införskaffande}
För mjukvara avsedd att utvecklas och/eller användas i sektionens namn gäller:
\begin{enumerate}
	\item Alla mjukvaruprojekt utvecklade ska vara  ``open source'', enl def. av Open Source Initiative, \MYhref{http://opensource.org/definition}{http://opensource.org/definition}, och därav publikt tillgänglig för läsning, kopiering och modifiering bl.a. 
	\item Vid införskaffande av mjukvara ska ett open source-alternativ användas i första hand istället för proprietär mjukvara
 	\item Mjukvara som behöver utvecklas ska i första hand utvecklas internt, dvs att någon på sektionen får jobbet att utveckla den istället för att beställa jobbet från utomstående part.
\end{enumerate}

\end{document}
