\documentclass[11pt, noincludeaddress, nopagination]{classes/cthit}
\usepackage{titlesec}
\usepackage{verbatimbox}
\usepackage{tabularx}

\titleformat{\paragraph}[hang]{\normalfont\normalsize\bfseries}{\theparagraph}{1em}{}
\titlespacing*{\paragraph}{0pt}{3.25ex plus 1ex minus 0.2ex}{0.7em}

\graphicspath{ {images/} }

\begin{document}

\title{Proposition om att höja maxantalet lekmannarevisorer}
\authors{styrIT}

\makeheadfoot%

\makesimpletitle

\subsection*{Bakgrund}
I nuläget är maxantalet lekmannarevisorer på sektionen begränsade till fyra stycken enligt stadgan, vi ser detta som en arbiträr gräns som inte fyller något syfte. 
Genom att öka maxantalet hoppas vi göra revisorernas arbete enklare då fler potentiellt kan dela på arbetsbördan, vilket kan öka attraktiviteten av att söka revisor på sektionen. 
\\\\
Med detta som bakgrund vill vi därför utöka maxantalet lekmannarevisorer i stadgan från fyra till åtta. 
\subsection*{Förslag}
Yrkar på:
\begin{att}
    \item ändra ''§14.1.1 inval'' \textbf{från:} \\
    Sektionsmötet utser två till \textbf{fyra} lekmannarevisorer med uppgift att granska teknolog-sektionens verksamhet och ekonomi under verksamhetsåret.\\
    \textbf{till:} \\
    Sektionsmötet utser två till \textbf{åtta} lekmannarevisorer med uppgift att granska teknolog-sektionens verksamhet och ekonomi under verksamhetsåret.
\end{att}

\end{document}
