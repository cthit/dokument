\documentclass[11pt, noincludeaddress, nopagination]{classes/cthit}
\usepackage{titlesec}
\usepackage{verbatimbox}
\usepackage{tabularx}

\titleformat{\paragraph}[hang]{\normalfont\normalsize\bfseries}{\theparagraph}{1em}{}
\titlespacing*{\paragraph}{0pt}{3.25ex plus 1ex minus 0.2ex}{0.7em}

\graphicspath{ {images/} }

\begin{document}

\title{Svar på: Motion om att starta kommittén FlashIT}
\authors{styrIT}

\makeheadfoot%

\makesimpletitle

\subsection*{Svar}
Vi håller med motionären om att FlashIT hade passat bättre som en kommitté än som en förening, FlashIT jobbar i nuläget nästan exklusivt mot sektionen. 
Utöver detta kan FlashIT som kommitté dra nytta av ett antal fördelar som en kommitté får från sektionen vilka en FlashIT som förening inte kunnat dra nytta av. 
Vi ser dock gärna att FlashITs uppdrag utökas till att inkludera främjande av fotograferingsintresse på sektionen, detta för att spegla föreningen FlashITs syfte.
Med detta som bakgrund står styrIT bakom motionen, däremot ser vi att i den nuvarande formulering saknas det några punkter som vi ser som viktiga, samt att några ändringar krävs på grund av formalia. 

\subsection*{Yrkanden}

\begin{att} 
\item specificera att den andra \textbf{att} satsen i motionen läggs till under ''§5.4 Kommittéspecifika åligganden''.
\item lägga till följande \textbf{att} satser under ''§5.4 Kommittéspecifika åligganden'':
\begin{att}
\item uppmuntra och bidra till ökat fotograferingsintresse på sektionen. 
\end{att}
\item på första ordinarie sektionsmöte LP1 2020 välja in en intermittent styrelse för FlashIT med uppdrag:
\begin{att}
\item starta upp kommittén fram till inval LP3.
\item att tillsammans med styrIT se över FlashITs ålägganden och uppdrag samt jobba med att lösa praktiska problem som kan uppkomma vid övergången till kommitté.
\item att om så efterfrågas förse sektionens organ med profileringsbilder likt hur kommitteen ska föra sin verksamhet i framtiden.
\end{att}
\item beslutet direktjusteras
\end{att}

\end{document}