\documentclass[11pt, noincludeaddress, nopagination]{classes/cthit}
\usepackage{titlesec}
\usepackage{verbatimbox}
\usepackage{tabularx}

\titleformat{\paragraph}[hang]{\normalfont\normalsize\bfseries}{\theparagraph}{1em}{}
\titlespacing*{\paragraph}{0pt}{3.25ex plus 1ex minus 0.2ex}{0.7em}

\graphicspath{ {images/} }

\begin{document}

\title{Möteshållning för sektionsmöte över Zoom}
\authors{styrIT}

\makeheadfoot%

\makesimpletitle

\vspace{-20pt}
\subsection*{Bakgrund}
Då Zoom som plattform inte ger oss samma förutsättningar som en hörsal behöver vi definiera en möteshållning för att vi ska kunna ta oss igenom sektionsmötet så smidigt som möjligt.

Härefter följer regler, rekommendationer och upplysningar som tillsammans uppgör möteshållningen. Dessa är framtagna för att förenkla sektionsmötet. Dokumentet skall användas tillsammans med \href{https://styrit.chalmers.it/wp-content/uploads/sektionsmoteshandbok.pdf}{sektionsmöteshandboken} där reglerna i detta dokument kommer ha företräde framför handboken.

% Ändring och/eller utökning av möteshållningen kan komma att ske fram till torsdag 8e oktober kl 17:00.

\subsection*{Allmänt}
\begin{itemize}
    
    \item Varje deltagare får endast vara ansluten på sektionsmötet med en enhet.
    \item Varje deltagare behöver vara ansluten genom sitt Chalmers-Zoomkonto (Chalmers.zoom.us), för att vi ska kunna identifiera att deltagare är IT-teknologer.
    \item Alla deltagare behöver vara anslutna till sin egen enhet och inte dela med någon annan.
    \item Då Zooms ''Poll'' funktion inte fungerar i webbläsaren rekommenderas närvarande på mötet att använda en Zoomklient (mobil- eller desktopapplikationen).
    \item Användarnamnet för varje deltagare måste innehålla deltagarens för- och efternamn, samt CID. Tillåtna namn är exempelvis:
    \begin{itemize}
        \item Eric Carlsson (caeric)
        \item Eric ''LP'' Carlsson (caeric)
    \end{itemize}
    Vid otydlighet innehar styrIT rätten att ändra deltagares namn.
    \item Sektionsmötet kommer ske i ett Breakout room, inbjudan dit sker av styrIT vid mötets starttid. Insläpp vid sen ankomst sker kontinuerligt.
    \item För allas trevnad uppmanas alla som har tillgång till kamera att använda denna.
\end{itemize}

\subsection*{Begäran av ordet}
\begin{itemize}
    \item Begäran av ordet sker via \href{https://talarlista.chalmers.it/user}{talarlista.chalmers.it}
    \item Begäran av ordet vid Sakupplysningar och Ordningsfrågor kan även göras genom att skriva direkt på Zoom, alternativ till någon i styrIT på Slack.
    \item Vid ändringsyrkanden kan man skriva direkt till någon i styrIT med yrkandet.
    % \item Replik kan begäras genom att skriva upp sig på talarlistan eller genom att skriva till någon i styrIT på Zoom eller slack. 
    \item Replik kan tilldelas direkt av mötets ordförande, vid behov kan det begäras genom att skriva direkt på Zoom, exempelvis \textit{Replik som ArmIT}.
\end{itemize}

\subsection*{Personinval}
Vid inval kommer de nominerade att placeras i ett breakout room tillsammans med en representant från styrIT (istället för att lämna rummet som vid ett fysiskt sektionsmöte). Vänligen respektera att du som nominerad inte ska vara vid hörselavstånd (fysiskt) till övriga mötesdeltagare vid dessa tillfällen.

\subsection*{Acklamation}
Acklamation kommer ske direkt genom en ''Yes'' (Grön checkbox) och ''No'' (Rött kryss) i Zoom.

\subsection*{Votering}
% Votering kommer främst ske med hjälp av Zooms ''Poll'' funktion. Annars kommer röstning att ske genom ''Yes'' (Grön checkbox) och ''No'' (Rött kryss) i Zoom. Vilket som används specificeras av mötets ordförande innan votering sker.  

Votering kommer att ske på ett av två alternativ:

\begin{itemize}
    \item Använding av Zooms ''Poll''. 
    \item Samma som acklamation, använding av ''Yes'' (Grön checkbox) och ''No'' (Rött kryss).
\end{itemize}

Vilket som används specificeras av mötets ordförande innan votering sker.  

\subsection*{Sluten votering}
Om sluten votering begärs kommer voteIT (ITs version) att användas. Koder kommer i förstahand mejlas av mötets justerare till de som är närvarande i Zoommötet. Sluten votering kan begäras genom talarlistan.

\end{document}
