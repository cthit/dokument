\documentclass[11pt, noincludeaddress, nopagination]{classes/cthit}
\usepackage{titlesec}
\usepackage{verbatimbox}
\usepackage{tabularx}

%  Verksamhetsrapporterna ska spegla vad ni gjort i er verksamhet sedan senaste ordinarie sektionsmöte, och är sektionens sätt att veta vad ni gör och att ni har gjort det ni är ålagda. 

% Filerna ska heta vrapport_kommitténamn_LPX_20XX.pdf utan punkter och versaler i kommittésnamnet, t.ex. "digit", "prit" eller "fritid". X ska vara siffror för läsperiod och årtal.  

% ÄNDRA HÄR
    \newcommand{\kommitte}{styrIT 20/21}
    \newcommand{\datum}{2020--21}
    \newcommand{\lp}{LP1}

% DETTA SKA INTE ÄNDRAS
\titleformat{\paragraph}[hang]{\normalfont\normalsize\bfseries}{\theparagraph}{1em}{}
\titlespacing*{\paragraph}{0pt}{3.25ex plus 1ex minus 0.2ex}{0.7em}
\graphicspath{ {images/} }
\begin{document}
\title{Verksamhetsplan \kommitte{}}
\authors{\kommitte}
\makeheadfoot%
\begin{center}
    \Huge{\textbf{Verksamhetsplan \kommitte{} }}
\end{center}


% ÄNDRA NEDAN TILL SAKER SOM STÄMMER FÖR ER
\section{Förord}
Styrelsens verksamhetsplan grundar sig i sektionens mål- och visionsdokument, det kontinuerliga arbetet och de fokusområden styrIT väljer att jobba med.
\\\\
På grund av Covid-19 finns det en relativt stor chans att delar av verksamheten inte kommer kunna genomföras som den brukar under året. Vi väljer dock att vara positiva och hoppas på att vårt och sektionens arbetet kommer kunna ske på ett någorlunda normalt sätt under året, vilket reflekteras i verksamhetsplanen.
\section{Operativt arbete}
De kontinuerliga åtaganden styrIT har kommer under året uppta den största delen av arbetet. Detta jobb består av att:
\begin{itemize}
    \item Delta på de olika utskott, arbetsgrupper samt övriga möten som styrIT skall vararepresenterade på.
    \item Behandla frågor rörande sektionen som helhet.
    \item Hantera incidenter.
    \item Utbilda samt hjälpa sektionsaktiva.
    \item Behandla inkomna äskningar.
    \item Vara delarrangör av Kandidatmiddagen.
    \item Arrangera sektionsmöten.
    \item Underhålla styrdokument.
    \item Arbeta för att sektionens medlemmar ska må så bra som möjligt.
    \item Arbeta för att inkludera sektionens medlemmar i sektionens verksamhet.
    \item Underhålla och uppdatera dokumentation relaterade till ekonomin.
    \item Utvärdera nöjeslivet på sektionen kontinuerligt.
    \item Jobba reaktivt med hur sektionen ska hantera Covid-19. 
\end{itemize}

\section{Fokusområden}
Förutom det kontinuerliga arbetet har styrIT'20 valt några fokusområden som vi ska jobba mer med under årets gång. Detta för att belysa vad vi ser som viktiga frågor kopplade till nöjeslivet, studiemiljön och sammanhållningen på sektionen. Förutom detta vill vi förbättra och utveckla områden av sektionens arbete som vi anser inte håller tillräckligt hög nivå. \\\\
Valet av fokusområden har grundat sig i rekommendationer från förra årets styrelse, åsikter från sektionen samt områden som framkommit under diskussion inom styrelsen. Fokusområdena presenteras inte i någon speciell ordning. 

\subsection{Hållbar expansion}
Grundprogrammet kommer under de två kommande åren att expandera kraftigt samtidigt som även masterprogrammen utökar antalet platser. Detta leder till att antalet medlemmar i sektionen kommer öka med ca 40\% till år 2025. Styrelsen vill därför redan nu börja jobba för att vara med i förhandlingarna med högskolan så att vi kan behålla eller förbättra studiemiljön och utbildningskvaliteten på sektionen och programmet. 

\subsection{Bättre lokaler och studiemiljö}
Enligt studentbarometern upplevs studiemiljön som väldigt dålig av studenterna på IT-sektionen, Tyvärr är detta ett problem som inte går att lösa enkelt. Styrelsen vill därför trycka mer på högskolan angående bristen på lokaler, samt på att studiemiljön i lokalerna som finns är dålig. Utöver detta vill styrelsen jobba med att utöka den till sektionen tillgängliga ytan i Hubben och färdigställa det sista arbetet med The Cloud.

\subsection{Förbättrad sammanhållning på sektionen}
Sammanhållningen på sektionen, framförallt mellan de som studerar på campus Johanneberg och de som studerar på Lindholmen är i dagsläget undermålig. Det kan finnas flera olika anledningar till detta och vanligt påtalat är den fysiska distansen mellanprogrammen. Det är också relativt få studenter från grundprogrammet som går vidare till de tillhörande masterprogrammen vilket vidare försvårar kontakten med masterprogrammen. Vi tror att sektionen behöver fortsätta arbeta med att säkerhetsställa att utbildningsinnehållet och kvaliteten på masterprogrammen överensstämmer med förväntningarna hos grundprogrammets studenter. 
\\\\
Under den här punkten vill vi även jobba med att det ska vara mer lockande för studenter som studerar på Lindholmen att vara med på arrangemang på Johanneberg, och med att nöjeslivet på Lindholmen ska bli mer attraktivt.

%\subsection{Visioner och utvecklingsprojekt}
%Sektionen jobbar i nuläget inte så mycket med utvecklingsprojekt som den kanske borde, mycket fokus ligger istället på reaktivt arbete. Styrelsen vill därför jobba mer med långsiktiga frågor %samt strategiskt arbete för att sektionen ska fungera bättre, inte bara från idag utan även om 5 år. 


%Skapa bättre alumnikontakt. 

%Google forms där man kan hjälpa till att bestämma vad styrelsen ska arbeta med.

\end{document}



