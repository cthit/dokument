\documentclass[11pt, noincludeaddress, nopagination]{classes/cthit}
\usepackage{titlesec}
\usepackage{verbatimbox}
\usepackage{tabularx}

\titleformat{\paragraph}[hang]{\normalfont\normalsize\bfseries}{\theparagraph}{1em}{}
\titlespacing*{\paragraph}{0pt}{3.25ex plus 1ex minus 0.2ex}{0.7em}

\graphicspath{ {images/} }

\begin{document}

\title{Proposition om att lägga till undantag i stadgan vid krig, pandemi eller annan katastrof}
\authors{styrIT}

\makeheadfoot%

\makesimpletitle

\subsection*{Bakgrund}
Som resultat av Covid-19 och restriktionerna från kåren, högskolan och regeringen har styrelsen vid tillfällen haft svårt att anpassa delar av sektionens verksamhet. Framförallt har styrelsen sett att det har funnits behov av att frångå de normala förfaranden när det kommer till sektionsmöten. I mer detalj hade styrelsen sett att det vid krig, pandemi eller annan katastrof fanns större flexibilitet kring att ställa in sektionsmöten och flytta i reglementet fastslagna punkter till senarkommande sektionsmöten. Styrelsen föreslår därför att lägga till ett kapitel ..... \textit{Extraordinära händelser} som hanterar detta. 
\subsection*{Förslag}
Yrkar på:
\begin{att}
    \item \textbf{Lägga till} nedanstående kapitel som kapitel 15 i stadgan
    \item \textbf{Numrera om} kapitel nuvarande kapitel 15 till 16, 
\end{att}
\newpage

\section*{§15~ Extraordinära händelser}
\subsection*{§15.1 Frångång av reglemente}
Vid extraordinära händelser så som krig, pandemier, naturkatastrofer som resulterar i att sektionens reglemente och policies inte kan följas kan sektionsstyrelsen frångå 


\end{document}
