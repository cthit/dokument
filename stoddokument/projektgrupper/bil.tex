\section{Projektgrupp bil}

\subsection{Uppdrag}
Denna projektgrupp, har i uppdrag att komma fram till en plan om hur införande av sektionsbil kan gå till.

Projektgruppen ges i uppdrag att undersöka och analysera minst följande aspekter:

\begin{itemize}
    \item Det praktiska behovet av en sektionsbil inom sektionen (nuvarande och framtida).
    \item Ekonomiska faktorer: Leasing eller inköp, uppskattade driftskostnader (bränsle, försäkring, skatt, service, reparationer), möjliga finansieringsmodeller och budgetpåverkan för sektionen.
    \item Praktiska och logistiska aspekter: Parkeringslösningar, system för bokning och nyckelhantering, rutiner för underhåll och skötsel, regler och riktlinjer för användning (vem får köra, för vilka syften, etc.).
    \item Juridiska aspekter och ansvarsfrågor: Försäkringsskydd, ansvar vid skada eller olycka.
    \item Miljöaspekter
\end{itemize}

\subsection{Tidsram}
Projektgruppen valeds in under ordinarie möte LP4 2025, och ska presentera sina samlade slutsatser, en analys av för- och nackdelar, samt en rekommendation (för eller emot införande, och i så fall en föreslagen modell för detta) för sektionsmötet senast vid ordinarie sektionsmöte LP 2 2025.
