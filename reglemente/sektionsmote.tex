\section{Sektionsmötet}

\subsection{Kallelse}
Kallelse till sektionsmöte består av mötets
\begin{itemize}
    \item datum
    \item tid
    \item plats.
\end{itemize}

Kallelsen skall publiceras väl synligt för sektionens medlemmar på sektionens anslagstavla, webbsida och sektionslokal. Kallelsen skall anslås korrekt enligt stadgan. 

\subsection{Möteshandlingar}
Förslag till dagordningen skall anslås fem dagar innan sektionsmötet och förslaget skall innehålla:

\begin{itemize}
	\item Datum, plats och tid för mötet
	\item Mötets öppnande
	\item Val av mötets ordförande
	\item Val av mötets sekreterare
	\item Val av mötets justerare tillika rösträknare
	\item Mötets behöriga utlysande
    \item Fastställande av mötets ordning
	\item Fastställande av mötets dagordning
	\item Adjungeringar
	\item Föregående mötesprotokoll
	\item Meddelanden
	\item Verksamhetsrapporter
	\item Eventuella års- och revisionsberättelser
	\item Interpellationer
    \item Frågor till mötet
    \item Propositioner
	\item Motioner
	\item Eventuella personval
	\item Övriga frågor
	\item Mötets avslutande
\end{itemize}

\subsection{Mötesordning}
Mötesordningen fastställs under sektionsmötet. Dessa bestämmelser ersätter inte bestämmelser i sektionens styrdokument.

\subsubsection{Reservation}
Reservation mot beslut av sektionsmötet skall anmälas skriftligen senast 24 timmar efter mötet.

\subsubsection{Ajournering}
Bifalles yrkandet om ajournering av mötesordförande skall tidslängden av ajourneringen fastställas.

\subsubsection{Interpellation}
Sektionsorgan måste svara på interpellationer som inkommit innan sista inlämningsdagen för motioner enligt stadgarna. Om varken interpellatören eller en representant för interpellatören framför interpellationen framförs den av mötesordförande.

\subsubsection{Motion}
Motion som är upptagen på dagordningen måste lyftas och föredragas av motionären eller någon annan på mötet med förslagsrätt, annars faller motionen. Sektionsstyrelsen skall lämna sitt utlåtande om motioner som inkommit innan sista inlämningsdagen för motioner enligt stadgarna.
