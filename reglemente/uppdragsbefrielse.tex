\section{Uppdragsbefrielse}
Uppdragsbefrielse är ett beslut fattat av sektionsmötet som innebär att en person med ansvarspost inom sektionsorgan anses ha fullgjort sitt ansvar för det gångna verksamhetsåret i enlighet med sektionens styrdokument.
\\\\
För att uppdragsbefrielse ska kunna beviljas krävs att:
\begin{itemize} 
    \item Verksamhets- och ekonomisk berättelse presenteras och godkänts av sektionsmötet,
    \item Ekonomisk redovisning har granskats av lekmannarevisorerna,
    \item Eventuella revisionsanmärkningar har hanterats,
    \item Inga väsentliga brister påvisats som skulle påverka befrielsen.
\end{itemize}
Uppdragsbefrielse beslutas av sektionsmötet, efter förslag från styrIT eller revisorer.
\\\\
När uppdragsbefrielse beviljats godkänner sektionsmötet att uppdraget avslutats korrekt. Därefter riktas ej ytterligare krav mot berörd person eller organ för det angivna verksamhetsåret, förutsatt att inte allvarliga fel framkommer i efterhand.