\section{Sektionsorgan}

\subsection{Åligganden}

\subsubsection{Aspning}
Det åligger alla sektionsorgan att arrangera minst två arrangemang vars syfte är att väcka intresse för och informera om organets verksamhet för nya sökande.

\subsection{Tillträde}
I det fall att ett sektionsorgan inte uppfyller minimum antal medlemmar som är definierat enligt sektionens reglemente eller stadgar så tillträder de sin mandatperiod
enligt styrdokumenten med något modifierade åligganden.
\begin{itemize}
    \item Det åligger sektionsorganet ifråga att senast inför nästkommande sektionsmöte ha minst ett aspningstillfälle med syfte att fyllnadsvälja så att sektionsorganet kommer upp till minimumgränsen.
    \item Det åligger sektionsorganet att i den mån som är möjligt, uppfylla sina ordinarie åligganden.
\end{itemize}
Utöver detta tillkommer ett krav på att minst en av de invalda ska inneha ansvarspost. 
Om det endast finns en invald ansvarspost så kan sektionsmötet besluta att ge denne fullmakt att ha ansvar för kommittén fram till nästkommande sektionsmöte när en ytterligare ansvarspost ska väljas in.
\subsection{Vakans eller frånvaro i ansvarspost}
Om en ansvarspost i ett sektionsorgan inte väljs in vid invalsmöte, ansvarar organet att internt eller i samråd med styrIT fördela ansvaret tillfälligt, fram tills fyllnadsval sker.
\\\\
Om en person med ansvarspost är oförmögen att genomföra sitt uppdrag under en längre period (t.ex. på grund av sjukdom), ska styrIT i samråd med berört organ:
\begin{itemize}
    \item Utse en tillfällig ersättare internt i organet om möjligt, eller
    \item Temporärt överta det praktiska ansvaret för funktionen tills annan lösning beslutats.
\end{itemize}